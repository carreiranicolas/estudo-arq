\documentclass{report}

% Pacotes para acentuação e formatação
\usepackage[utf8]{inputenc}
\usepackage[T1]{fontenc}
\usepackage[brazil]{babel}
\usepackage{setspace}    % Para espaçamento
\usepackage{lipsum}      % Texto de exemplo (remova se não precisar)

\begin{document}
	
	\begin{titlepage}
	\centering
	\vspace*{5cm} % Espaço do topo
	
	{\Huge\bfseries Arquitetura de Computadores I (LAB)\par} % Título
	
	\vspace{0.5cm}
	{\Large 2025/2\par} % Ano
	
	\vfill
	{\large Nicolas Ramos Carreira\par} % Nome
	
	\vspace*{2cm}
	\end{titlepage}
	
	\tableofcontents
	\newpage
	
	\chapter{Intuito}
	O intuito deste documento é explicar sobre o que será feito nas aulas de laboratório da disciplina de Arquitetura de Computadores I. Neste documento provavelmente estarão explicações de componentes elétricos e alguns conceitos físicos, então se prepare!
	\chapter{Fundamentos Fisicos da Computação}
	\section{Carga elétrica}
	\section{Corrente elétrica}
	\section{Tensão(voltagem)}
	\section{Resistência}
	\section{Analogia(para um entendimento claro)}
	\section{Lei de Ohm}
	\section{Circuitos elétricos}
	\section{Lei de Kirchhoff}
	
	\chapter{Cicuito real}

\end{document}