\documentclass[12pt,a4paper]{report}

% Pacotes para acentuação e formatação
\usepackage[utf8]{inputenc}
\usepackage[T1]{fontenc}
\usepackage[brazil]{babel}
\usepackage{setspace}    % Para espaçamento
\usepackage{lipsum}      % Texto de exemplo (remova se não precisar)

\begin{document}
	
	% ----------- CAPA -----------
	\begin{titlepage}
		\centering
		\vspace*{5cm} % Espaço do topo
		
		{\Huge\bfseries Arquitetura de Computadores I\par} % Título
		
		\vspace{0.5cm}
		{\Large 2025/2\par} % Ano
		
		\vfill
		{\large Nicolas Ramos Carreira\par} % Nome
		
		\vspace*{2cm}
	\end{titlepage}
	
	% ----------- SUMÁRIO -----------
	\tableofcontents
	\newpage
	
	% ----------- CONTEÚDO -----------
	\chapter{Importância da matéria}
	Esta é sem dúvidas uma das disciplinas mais importantes para um Cientista da Computação, pois sem ela:
	
	\begin{itemize}
		\item Não saberemos como o computador funciona de fato
		\item Não seremos programadores tão bons como podemos ser
		\item Seremos mais sucessetiveis a cometer determinados erros
		
	\end{itemize}
	% Exemplo com texto fictício:
	%\lipsum[1]
	
	\chapter{O que aprenderemos}
	\section{Computadores são burros para fazer cálculos}
	\subsection{A conta: (43,1 - 43,2) + 1}
	\subsection{Propriedades matemáticas}
	\section{Como o Software roda no Hardware}
	
	\section{A memória}
	\subsection{O que é de fato?}
	\subsection{Bugs no referenciamento de memória são perniciosos}
	\subsection{O porquê entender}
	
\end{document}
